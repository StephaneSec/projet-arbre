
\documentclass[8pt]{beamer}
\usepackage[utf8]{inputenc}
\usepackage[T1]{fontenc}
\usepackage[french]{babel}



\AtBeginSection[]
{
  \begin{frame}
  \frametitle{\textcolor{white}{Contents}}
  \tiny{\tableofcontents[currentsection]}
  \end{frame}
}
\newtheorem*{remark}{Remark}
\newtheorem*{pb}{Problem}
\newtheorem*{sol}{Solution}
\newtheorem*{obj}{Objectives}



%\usetheme{Frankfurt} % theme
%\usetheme{Szeged} % theme
\usetheme{Ilmenau} % theme
%\usetheme{Darmstadt} % theme
%\usetheme{Montpellier} % theme
%\usetheme{Luebeck} % theme
%\usetheme{Antibes} % theme

\usepackage[footheight=1em]{beamerthemeboxes}




\titlegraphic{\includegraphics [width=\linewidth]{logos.png}}

\title[Graphs' decompositions\\ and \\ resolutions of combinatorial problems]{Graphs' decompositions\\ and \\ resolutions of combinatorial problems}
\author[Stéphane Secouard]{\small {Stéphane Secouard\\ \footnotesize Supervised by: Florent Madeleine}} 
\institute[Caen University]{\textbf {Caen University - Computer science }}
\date{25 october 2016}





\usepackage{tikz}
\usetikzlibrary{shapes,snakes}


 
\addfootboxtemplate{\color{black}}{\color{white}
     \hfill\insertframenumber/\inserttotalframenumber\hspace{2em}\null}
\begin{document}

%--------------------- diapo 1 ---------------------------------------
\begin{frame}

  \titlepage
\end{frame}

%---------------------- diapo 2 --------------------------------------

\begin{frame}

  \frametitle{ \textcolor{white}{Table of contents}}

  \tableofcontents

\end{frame}




%----------------- diapo 4 --------------------------------------------



\section{Objectives}

\begin{frame}
  \frametitle{ \textcolor{white}{Objectives}}

  \begin{obj}[of my project]
  \begin{itemize}
  \item The first goal of this project is, starting from a graph whose  tree representation is known, to solve corresponding combinatorial problems.

    The k-coloring problem, the max clique problem or the Hamilton path problem can be explored.

\item The second purpose is to implement a graph decomposition calculator.

\item Finally, it can be considered extensions by working on the efficiency of implementations on large-size structures, or improving the shape of displayed results.
  \end{itemize}
  \end{obj}

  \begin{obj}[of my presentation]
    Now, I am going to present :
    \begin{itemize}
    \item important concepts to understand the goals of my project;
    \item an example where these concepts help to solve a problem of coloration.
    \end{itemize}
    \end{obj}
\end{frame}



%------------------- diapo 6 -----------------------------------------
\section{Tree decomposition of a graph}

\subsection{definition}
\begin{frame}
\frametitle{\textcolor{white}{Tree decomposition of a graph}}  



    \begin{definition}[Tree decomposition of a graph]
  A tree T is a tree decomposition of a graph G  where its nodes are arranged satisfying the following properties :
  \begin{enumerate}
  \item If u and v are neighbors in G, then there is a bag of T containing both of them (a bag is a node of the tree).
  \item For every vertex v of G, the bags of T containing v form a connected subtree
  \end{enumerate}
\end{definition}


\uncover<2->{\underline{\textbf{Example :}}}

\begin{columns}
\begin{column}{0.46\textwidth}
  \begin{tikzpicture}[scale=0.8]
  \uncover<2->{ \node[draw,rectangle] (G) at (-1,1.5) {\textbf{(G)}};
  \node[draw,circle] (1) at (0,2) {1};
  \node[draw,circle] (2) at (0,1) {2};
  \node[draw,circle] (3) at (1,1.5) {3};
  \node[draw,circle] (4) at (2,2) {4};
  \node[draw,circle] (5) at (2,1) {5};
  \node[draw,circle] (6) at (3,2) {6};
  \node[draw,circle] (7) at (3,1) {7};
  \node[draw,circle] (8) at (4.5,2) {8};
  \node[draw,circle] (9) at (4.5,1.3) {9};
  \node[draw,circle] (10) at (4,0.5) {{\tiny 10}};
  \node[draw,circle] (11) at (4.8,0.5) {{\tiny 11}};
  \draw(1)--(3)--(2) (3)--(4)--(5)--(3) (4)--(6)--(5)--(7)--(4) (6)--(8)--(9)--(6)--(7)--(10)--(11);}
\end{tikzpicture}
\end{column}
\begin{column}{0.54\textwidth}
  \begin{tikzpicture}[scale=0.9]
    \uncover<2->{\draw[thick](-1,0)--(-1,2);}
    \only<3>{\node[draw,rectangle](T) at (0,2){\textbf{($T_1$)}};}
  \uncover<3>{ 
    \node[draw,ellipse](a) at (0,0){1,3};
    \node[draw,ellipse](b) at (2,0){2,3};
    \node[draw,ellipse](c) at (1,1){3,4,5};
    \node[draw,ellipse](d) at (3,2){4,5,6,7};
    \node[draw,ellipse](e) at (5,1){6,8,9};
    \node[draw,ellipse](f) at (3.3,1){7,10};
    \node[draw,ellipse](g) at (3.6,0){10,11};
    \draw(a)--(c)--(b) (c)--(d)--(e) (d)--(f)--(g);}
  
    \only<4>{\node[draw,rectangle](T) at (0,2){\textbf{($T_2$)}};}
  \uncover<4>{\node[draw,ellipse](h) at (1.5,1.5){4,5,6,7,8,9};
  \node[draw,ellipse](i) at (0,0){1,2,3};
  \node[draw,ellipse](j) at (3,0){7,10,11};
  \draw(j)--(h)--(i);}

      \only<5>{\node[draw,rectangle](T) at (0,2){\textbf{($T_3$)}};}
  \uncover<5>{\node[draw,ellipse](k) at (2,1){1,2,3,4,5,6,7,8,9,10,11};}
  
  \end{tikzpicture}
  
\end{column}
\end{columns}



\end{frame}



%-------------------- diapo 7 -----------------------------------------

\subsection{treewidth}


\begin{frame}
  \frametitle{\textcolor{white}{Treewidth}}

  
\begin{definition}[treewidth]
 \begin{itemize}
 \item The \emph{width} of a decomposition is (largest bag size - 1).
 \item The \emph{treewidth} of a graph is the lowest width of all decompositions of this graph.
  \end{itemize}
  

\end{definition}

\uncover<2->{\underline{\textbf{Example :}} Assume (T) is one of the  representations of (G) which are giving the treewidth of (G) :
  \begin{tabular}{ll}
    \begin{tikzpicture}[scale=0.8]
      \node[draw,rectangle] (G) at (3,2.8) {\textbf(G)};
      \node[draw,circle] (a) at (1,3) {a};
      \node[draw,circle] (b) at (0,2) {b};
      \node[draw,circle] (c) at (2,2) {c};
      \node[draw,circle] (d) at (4,2) {d};
      \node[draw,circle] (e) at (0,0) {e};
      \node[draw,circle] (f) at (2,0) {f};
      \node[draw,circle] (g) at (4,0) {g};
      \node[draw,circle] (h) at (6,0) {h};
      \draw (c)--(a)--(b)--(e)--(f)--(c)--(b);
      \draw (c)--(d)--(f)--(g)--(d)--(h)--(g);
    \end{tikzpicture}
    &
    \begin{tikzpicture}
      \draw[thick](-1,0)--(-1,2);
            \node[draw,rectangle] (T) at (3.8,1.6) {\textbf(T)};
            \node[draw,ellipse] (cdf) at (2,2) {c,d,f};
            \node[draw,ellipse] (bcf) at (1,1) {b,c,f};
            \node[draw,ellipse] (abc) at (0,0) {a,b,c};
            \node[draw,ellipse] (bef) at (2,0) {b,e,f};
            \node[draw,ellipse] (dfg) at (3,1) {d,f,g};
            \node[draw,ellipse] (dgh) at (4,0) {d,g,h};
            \draw (abc)--(bcf)--(cdf)--(dfg)--(dgh);
            \draw (bef)--(bcf);
    \end{tikzpicture}
    
  \end{tabular}
}
\uncover<3->{
  \vspace{-2mm}
  \begin{center}\fbox{Treewidth(G)=3-1=2}\end{center}}

%\uncover<4->{
%  \vspace{-2mm}
%  \begin{remark}
%    \begin{itemize}
%    \item The treewidth   of a tree is 1.
%    \item If a graph has a treewidth of 1 we can state that this graph is a forest (i.e. a collection of trees).
%    \end{itemize}
%    \end{remark}
%  }

\end{frame}



%----------------- diapo 8 --------------------------------------------

\subsection{nice tree}

\begin{frame}
  \frametitle{ \textcolor{white}{Nice tree}}
  
  \begin{definition}[nice tree]
    A tree decomposition is \emph{nice} if every node $x$ is one of the following 4 types  :
    \begin{description}
    \item[Leaf :] no children, $|B_x|=1$ \hfill ($B_x$ names a bag of the tree containing $x$)
    \item[Introduce :] 1 child $y$, $B_x=B_y\cup \{v\}$ for some vertex $v$
    \item[Forget :] 1 child $y$, $B_x=B_y\backslash \{v\}$ for some vertex $v$.
    \item[Join :] 2 children $y_1$, $y_2$ with $B_x=B_{y_1}=B_{y_2}$
    \end{description}

    \bigskip
    \begin{tabular}{c|c|c|c}
        
      Leaf & Introduce & Forget & Join\\
        
      \begin{tikzpicture}
        \node[draw,ellipse] (v) at (0,0) {\hspace{1mm}$v$\hspace{1mm}\ };
        \draw(-1,-1) (1,-1);
      \end{tikzpicture}
      &
      \begin{tikzpicture}
        \node[draw,ellipse] (u) at (0,0) {u, {\textcolor{red}{v}}, w };
        \node[draw,ellipse] (v) at (0,-1) {v,w};
        \draw(u)--(v);
        \draw(-1,-1) (1,-1);
      \end{tikzpicture}
      &
      \begin{tikzpicture}
        \node[draw,ellipse] (u) at (0,0) {u, w};
        \node[draw,ellipse] (v) at (0,-1) {u, {\textcolor{red}{v}}, w };
        \draw(u)--(v);
        \draw(-1,-1) (1,-1);
      \end{tikzpicture}
      &
      \begin{tikzpicture}
        \node[draw,ellipse] (u) at (0,0) {u, v, w};
        \node[draw,ellipse] (v) at (-1,-1) {u, v, w};
        \node[draw,ellipse] (w) at (1,-1) {u, v, w};
        \draw(v)--(u)--(w);
      \end{tikzpicture}
      
    \end{tabular}
  \end{definition}

  \begin{remark}
    \begin{itemize}
    \item A tree decomposition can be turned into a nice tree decomposition
    \item A nice tree can be used to simplify a proof, or to find an easy program to solve a problem (as we will see later on).
    \end{itemize}
  \end{remark}
  

\end{frame}


%----------------- diapo 9 -------------------------------------------
\subsection{Example}

\begin{frame}
  \frametitle{ \textcolor{white}{Example}}
  This is a tree decomposition of the graph seen previously. How can we get a nice tree from it ?

  \begin{columns}
    \begin{column}{0.3\textwidth}
      \begin{tikzpicture}
        \only<1->\node[draw,ellipse,fill=red] (cdf) at (2,2) {c,d,f};
        \only<2->\node[draw,ellipse,fill=green] (cdf) at (2,2) {c,d,f};
        
        \only<1-8>\node[draw,ellipse,fill=red] (bcf) at (1,1) {b,c,f};
        \only<8->\node[draw,ellipse,fill=green] (bcf) at (1,1) {b,c,f};
        
        \only<1-14>\node[draw,ellipse,fill=red] (abc) at (0,0) {a,b,c};
        \only<14->\node[draw,ellipse,fill=green] (abc) at (0,0) {a,b,c};
        
        \only<1-14>\node[draw,ellipse,fill=red] (bef) at (2,0) {b,e,f};
        \only<14->\node[draw,ellipse,fill=green] (bef) at (2,0) {b,e,f};
        
        \only<1-8>\node[draw,ellipse,fill=red] (dfg) at (3,1) {d,f,g};
        \only<8->\node[draw,ellipse,fill=green] (dfg) at (3,1) {d,f,g};
        
        \only<1-14>\node[draw,ellipse,fill=red] (dgh) at (4,0) {d,g,h};
        \only<14->\node[draw,ellipse,fill=green] (dgh) at (4,0) {d,g,h};
        \draw (abc)--(bcf)--(cdf)--(dfg)--(dgh);
        \draw (bef)--(bcf);
      \end{tikzpicture}
    \end{column}
    \begin{column}{0.5\textwidth}
      \begin{tikzpicture}[yscale=0.8]
        \uncover<18->{\node[draw,ellipse] (a) at (1/6,0) {a};}
        \uncover<16->{\node[draw,ellipse] (ab) at (2/6,1) {a,b};}
        \uncover<14->{\node[draw,ellipse] (abc) at (3/6,2) {a,b,c};}
        \uncover<12->{\node[draw,ellipse] (bc) at (4/6,3) {b,c};}
        \uncover<10->{\node[draw,ellipse] (bcf1) at (4/6,4) {b,c,f};}
        \uncover<8->{\node[draw,ellipse] (bcf) at (8/6,5) {b,c,f};}
        \uncover<10->{\node[draw,ellipse] (bcf2) at (12/6,4) {b,c,f};}
        \uncover<6->{\node[draw,ellipse] (cf) at (10/6,6) {c,f};}
        \uncover<4->{\node[draw,ellipse] (cdf1) at (11/6,7) {c,d,f};}
        \uncover<2->{\node[draw,ellipse] (cdf) at (14/6,8) {c,d,f};}
        \uncover<4->{\node[draw,ellipse] (cdf2) at (18/6,7) {c,d,f};}        
        \uncover<18->{\node[draw,ellipse] (b) at (14/6,0) {b};}
        \uncover<16->{\node[draw,ellipse] (be) at (13/6,1) {b,e};}
        \uncover<14->{\node[draw,ellipse] (bef) at (12/6,2) {b,e,f};}
        \uncover<12->{\node[draw,ellipse] (bf) at (11/6,3) {b,f};}
        \uncover<18->{\node[draw,ellipse] (d) at (23/6,1) {d};}
        \uncover<16->{\node[draw,ellipse] (dh) at (22/6,2) {d,h};}
        \uncover<14->{\node[draw,ellipse] (dgh) at (21/6,3) {d,g,h};}
        \uncover<12->{\node[draw,ellipse] (dg) at (20/6,4) {d,g};}
        \uncover<8->{\node[draw,ellipse] (dfg) at (19/6,5) {d,f,g};}
        \uncover<6->{\node[draw,ellipse] (df) at (18/6,6) {d,f};}
        \uncover<4->{\draw(cdf2)--(cdf)--(cdf1);}
        \uncover<6->{\draw(cdf1)--(cf) (cdf2)--(df);}
        \uncover<8->{\draw(cf)--(bcf) (df)--(dfg);}
        \uncover<10->{\draw(bcf)--(bcf1) (bcf)--(bcf2);}
        \uncover<12->{\draw(bcf1)--(bc) (bcf2)--(bf) (dfg)--(dg);}
        \uncover<14->{\draw(bc)--(abc) (bf)--(bef) (dg)--(dgh);}
        \uncover<16->{\draw(abc)--(ab) (bef)--(be) (dgh)--(dh);}
        \uncover<18->{\draw(ab)--(a) (be)--(b) (dh)--(d);}
        \uncover<3-4>{\node[draw,rectangle] (a) at (-1,7) {\textcolor{red}{JOIN}};}
        \uncover<5->{\node[draw,rectangle,black!20!white] (a) at (-1,7) {\textcolor{red!30!white}{JOIN}};}
        \uncover<5-6>{\node[draw,rectangle] (a) at (-1,6) {\textcolor{red}{INTRODUCE }};}
        \uncover<7->{\node[draw,rectangle,black!20!white] (a) at (-1,6) {\textcolor{red!30!white}{INTRODUCE }};}
        \uncover<7-8>{\node[draw,rectangle] (a) at (-1,5) {\textcolor{red}{FORGET}};}
        \uncover<9->{\node[draw,rectangle,black!20!white] (a) at (-1,5) {\textcolor{red!30!white}{FORGET}};}
        \uncover<9-10>{\node[draw,rectangle] (a) at (-1,4) {\textcolor{red}{JOIN}};}
        \uncover<11->{\node[draw,rectangle,black!20!white] (a) at (-1,4) {\textcolor{red!30!white}{JOIN}};}
        \uncover<11-12>{\node[draw,rectangle] (a) at (-1,3) {\textcolor{red}{INTRODUCE }};}
        \uncover<13->{\node[draw,rectangle,black!20!white] (a) at (-1,3) {\textcolor{red!30!white}{INTRODUCE }};}
        \uncover<13-14>{\node[draw,rectangle] (a) at (-1.2,2) {\textcolor{red}{FORGET}};}
        \uncover<15->{\node[draw,rectangle,black!20!white] (a) at (-1.2,2) {\textcolor{red!30!white}{FORGET}};}
        \uncover<15-16>{\node[draw,rectangle] (a) at (-1.5,1) {\textcolor{red}{INTRODUCE }};}
        \uncover<17->{\node[draw,rectangle,black!20!white] (a) at (-1.5,1) {\textcolor{red!30!white}{INTRODUCE }};}
        \uncover<17-18>{\node[draw,rectangle] (a) at (-1.5,0.3) {\textcolor{red}{INTRODUCE }};}
        \uncover<19->{\node[draw,rectangle,black!20!white] (a) at (-1.5,0.3) {\textcolor{red!30!white}{INTRODUCE }};}
        \uncover<19>{\node[draw,rectangle] (a) at (-1,-0.4) {\textcolor{red}{LEAF}};}
        
      \end{tikzpicture}
    \end{column}
  \end{columns}
  
\end{frame}








%----------------- diapo 11 --------------------------------------------



\section{Application : k-color}
\subsection{the problem}

\begin{frame}
  \frametitle{ \textcolor{white}{the problem}}
\begin{pb}[k-color]
Let (G) be a graph and k an integer. We want to know if it is possible to draw each vertice of the graph so that two neighbors have never the same color and with only k colors.

This problem is a problem of decisions problem which is NP-complet.
  \end{pb}

  \begin{sol}
  \begin{description}
  \item[tree-width :] k-color is possible for a graph (G) if and only if $k> treewidth(G)$.
  \item[nice tree :] a nice tree of (G) gives a way to find a k-coloration of (G) (if $k> treewidth(G)$).
  \end{description}
\end{sol}


\uncover<2>{
  \underline{\textbf{Illustration :}} we will use the previously trees to solve the problem with this graph :
  \begin{center}  
  \begin{tikzpicture}[scale=0.8]
    \node[draw,circle] (a) at (1,3) {a};
    \node[draw,circle] (b) at (0,2) {b};
    \node[draw,circle] (c) at (2,2) {c};
    \node[draw,circle] (d) at (4,2) {d};
    \node[draw,circle] (e) at (0,0) {e};
    \node[draw,circle] (f) at (2,0) {f};
    \node[draw,circle] (g) at (4,0) {g};
    \node[draw,circle] (h) at (6,0) {h};
    \draw (c)--(a)--(b)--(e)--(f)--(c)--(b);
    \draw (c)--(d)--(f)--(g)--(d)--(h)--(g);
  \end{tikzpicture}      
\end{center}
}

\end{frame}










%----------------- diapo 12 --------------------------------------------
\renewcommand\a{\textcolor{red}{a}}
\renewcommand\b{\textcolor{blue}{b}}
\renewcommand\c{\textcolor{green}{c}}
\renewcommand\d{\textcolor{blue}{d}}
\newcommand\e{\textcolor{green}{e}}
\newcommand\f{\textcolor{red}{f}}
\newcommand\g{\textcolor{green}{g}}
\newcommand\h{\textcolor{red}{h}}

\subsection{illustration}
\begin{frame}

    \frametitle{ \textcolor{white}{nice tree of the graph}}
    \begin{tabular}{ll}
\begin{minipage}{0.5\textwidth}
      \begin{itemize}
      \item treewith(G)=2 : we can solve the problem with 3 colors
      \item we can fix a color for a ``Leaf''
      \item when we meet an ``Introduce'' we add a color
      \item when we meet a ``Forget'' we can state that the vertex which has disappeared won't come back (property of the decomposition) and so we can reuse its color.
        
        \end{itemize}
      
      \begin{tikzpicture}[scale=0.8]
  \node[draw,circle] (a) at (1,3) {a};
  \node[draw,circle] (b) at (0,2) {b};
  \node[draw,circle] (c) at (2,2) {c};
  \node[draw,circle] (d) at (4,2) {d};
  \node[draw,circle] (e) at (0,0) {e};
  \node[draw,circle] (f) at (2,0) {f};
  \node[draw,circle] (g) at (4,0) {g};
  \node[draw,circle] (h) at (6,0) {h};
  \draw (c)--(a)--(b)--(e)--(f)--(c)--(b);
  \draw (c)--(d)--(f)--(g)--(d)--(h)--(g);
  \only<23->\node[draw,circle,fill=red] (a) at (1,3) {a};
  \only<23-> \node[draw,circle,fill=blue] (b) at (0,2) {b};
  \only<23-> \node[draw,circle,fill=green] (c) at (2,2) {c};
  \only<23-> \node[draw,circle,fill=blue] (d) at (4,2) {d};
  \only<23-> \node[draw,circle,fill=green] (e) at (0,0) {e};
  \only<23-> \node[draw,circle,fill=red] (f) at (2,0) {f};
  \only<23-> \node[draw,circle,fill=green] (g) at (4,0) {g};
  \only<23-> \node[draw,circle,fill=red] (h) at (6,0) {h};
  \draw(0,-12);

      \end{tikzpicture}
      \end{minipage}
      &
\begin{tikzpicture}[yscale=0.85]
  \node[draw,ellipse] (a) at (1/6,0) {a};
  \only<2->\node[draw,ellipse] (a) at (1/6,0) {\a};
  
  \node[draw,ellipse] (ab) at (2/6,1) {a,b};
  \only<3->\node[draw,ellipse] (ab) at (2/6,1) {\a,\b};
  
  \node[draw,ellipse] (abc) at (3/6,2) {a,b,c};
  \only<4->\node[draw,ellipse] (abc) at (3/6,2) {\a,\b,\c};
  
  \node[draw,ellipse] (bc) at (4/6,3) {b,c};
  \only<5->\node[draw,ellipse] (bc) at (4/6,3) {\b,\c};
  
  \node[draw,ellipse] (bcf1) at (4/6,4) {b,c,f};
  \only<6->\node[draw,ellipse] (bcf1) at (4/6,4) {\b,\c,\f};
  
  \node[draw,ellipse] (bcf) at (8/6,5) {b,c,f};
  \only<7->\node[draw,ellipse] (bcf) at (8/6,5) {\b,\c,\f};
  
  \node[draw,ellipse] (bcf2) at (12/6,4) {b,c,f};
  \only<8->\node[draw,ellipse] (bcf2) at (12/6,4) {\b,\c,\f};
  
  \node[draw,ellipse] (cf) at (10/6,6) {c,f};
  \only<13->\node[draw,ellipse] (cf) at (10/6,6) {\c,\f};
  
  \node[draw,ellipse] (cdf1) at (11/6,7) {c,d,f};
  \only<14->\node[draw,ellipse] (cdf1) at (11/6,7) {\c,\d,\f};
  
  \node[draw,ellipse] (cdf) at (14/6,8) {c,d,f};
  \only<15->\node[draw,ellipse] (cdf) at (14/6,8) {\c,\d,\f};
  
  \node[draw,ellipse] (cdf2) at (18/6,7) {c,d,f};
  \only<16->\node[draw,ellipse] (cdf2) at (18/6,7) {\c,\d,\f};
  
  \node[draw,ellipse] (b) at (14/6,0) {b};
  \only<12->\node[draw,ellipse] (b) at (14/6,0) {\b};
  
  \node[draw,ellipse] (be) at (13/6,1) {b,e};
  \only<11->\node[draw,ellipse] (be) at (13/6,1) {\b,\e};
  
  \node[draw,ellipse] (bef) at (12/6,2) {b,e,f};
  \only<10->\node[draw,ellipse] (bef) at (12/6,2) {\b,\e,\f};
  
  \node[draw,ellipse] (bf) at (11/6,3) {b,f};
  \only<9->\node[draw,ellipse] (bf) at (11/6,3) {\b,\f};
  
  \node[draw,ellipse] (d) at (23/6,1) {d};
  \only<22->\node[draw,ellipse] (d) at (23/6,1) {\d};
  
  \node[draw,ellipse] (dh) at (22/6,2) {d,h};
  \only<21->\node[draw,ellipse] (dh) at (22/6,2) {\d,\h};
  
  \node[draw,ellipse] (dgh) at (21/6,3) {d,g,h};
  \only<20->\node[draw,ellipse] (dgh) at (21/6,3) {\d,\g,\h};
  
  \node[draw,ellipse] (dg) at (20/6,4) {d,g};
  \only<19->\node[draw,ellipse] (dg) at (20/6,4) {\d,\g};
  
  \node[draw,ellipse] (dfg) at (19/6,5) {d,f,g};
  \only<18->\node[draw,ellipse] (dfg) at (19/6,5) {\d,\f,\g};
  
  \node[draw,ellipse] (df) at (18/6,6) {d,f};
  \only<17->\node[draw,ellipse] (df) at (18/6,6) {\d,\f};
  
      \draw(cdf2)--(cdf)--(cdf1);
      \draw(cdf1)--(cf) (cdf2)--(df);
      \draw(cf)--(bcf) (df)--(dfg);
      \draw(bcf)--(bcf1) (bcf)--(bcf2);
      \draw(bcf1)--(bc) (bcf2)--(bf) (dfg)--(dg);
      \draw(bc)--(abc) (bf)--(bef) (dg)--(dgh);
      \draw(abc)--(ab) (bef)--(be) (dgh)--(dh);
      \draw(ab)--(a) (be)--(b) (dh)--(d);
\draw(-1,-2);
\end{tikzpicture}
\\\end{tabular}
\end{frame}



%----------------- diapo 13 --------------------------------------------




%\subsection{max clique}
%\subsection{Hamiltonian path problem}

%\begin{frame}
%  \frametitle{ \textcolor{white}{Other applications}}

%  Many other problems can be study with a tree decomposition. For examples, we could have a look for :
%  \begin{description}
%  \item[Clique problem :] A clique in a graph is a complete subgraph (i.e. where any node pair is connected by an edge).

%    The Clique problem is how to find a maximal clique in a graph.
%  \item[Hamilton path problem :] a Hamiltonian path  is a path in a graph that visits each vertex exactly once.

%    A Hamiltonian cycle is a Hamiltonian path that ends at the starting point.
    
%    Determining whether such paths and cycles exist in a graph is the Hamiltonian path problem, which is NP-complete.
%    \end{description}
%\end{frame}






%----------------- diapo 14--------------------------------------------



\section{Bibliography}

\begin{frame}
  \frametitle{ \textcolor{white}{Bibliography}}

  \begin{description}
  \item[(1)] Florent Madeleine : lesson for M2 Decim "Complexité des CSP et des requêtes"
  \item[(2)] Dániel Marx : Fixed Parameter Algorithms
  \item[(3)] wikipedia : articles of the graph section
    \end{description}
\end{frame}
\end{document}
 
