\documentclass[a4paper, 10pt,french,landscape]{article}
\usepackage[dvips,top=1.5cm, bottom=1.5cm, left=1.5cm, right=1.5cm]{geometry}
\usepackage{PreambuleXelatex}
\usepackage{MacroTikz}
\usepackage{MacroDoc}
\pagestyle{empty}

\setlength{\parindent}{0em}


\begin{document}
\begin{multicols}{3}

\subsubsection*{ 1 }

Hello,
my project for this year is named "graphs' decompositions and resolutions of combinatorial problems" 
and this one is supervised by Florent Madeleine

\subsubsection*{ 2}

Here is the scheeme of my presentation. 
First, I will briefly expose the objectives of my project

Secondly I will introduce you three concepts : tree decomposition, treewidth and nice tree.

Finally, We will see an example of a problem's resolution which is using a nice tree and the treewidth.


\subsubsection*{ 3-4}





This project deals with a type of tree which is representing graph.

 A graph is more complex than a tree but sometimes it is possible to reduce it with a tree representation to solve problems because a tree is often easier to manipulate.

That is the reason why my first task will be to study some classicals' problems where there is a graph by using a tree representation of it.

After that, I will have to implement a program to solve some of these problems.

Once this first step is done, I will have to implement a program which is able to give a tree representation of a graph.

Now, I am going to expose some concepts which are useful in my project.



\subsubsection*{ 5-6}




A graph is not always as simple as a tree but it's possible to resume it  in a particular tree while keeping some graph's informations into the tree. 

To do that, the nodes of the tree represent a collection of nodes of the graph such that : 

* for example, the node 3 and the node 5 are connected with a edge and we can see that there is a bag which is containing both of them

* for example, the node 3 is in three bags in the right's tree which are forming a sub-tree.

It's important to realise two facts : 

* lots of  trees can represent a same graph
* all graph has at least one tree representation : T3 is just a leaf but it's always possible to construct such a tree representation.


\subsubsection*{ 7}

%The more a graph is strange (for example with a lot of loops), the more its tree representation has to contain some big node. 

It seams important to determine how far is a graph from a tree : this is the role of the treewidth

* the first point defines the width of a decomposition
* the second point defines the treewidth of a graph which is the width of the ``best'' tree decomposition (where the width is minimum).


In the example, I assume that T is one of the best tree decomposition of G.
Each bag's size is 3 so the treewidth is 2.


%We can understand that we want to consider the tree with the smallest nodes. 

%In the example, each bag of T has a width of 3 so the treewidth is 2. 

%In comparaison, the graph of the previous slide has a treewidth of 3 and it seams obvious (at least to me) that its structure is more far from a tree than the present graph.



\subsubsection*{ 8}

Now that we can evaluate the treewidth of a graph, I will expose a particular tree representation : the nice  tree

In a nice tree, we can only find this four situations : 

Leaf : no children

Introduce and Forget: in a bottom-up approach, there is one lettre more or one lettre less.

Join : in a bottom-up approach, two same nodes give a unique node with the same content.

The remark will be more explained in the two next slides. 

%It serves two purposes : 

%* can we always obtain a nice tree ? The response is yes and we will show you in the next slide
%* why is it useful ? It's to decompose complex mechanisms and thereby to answer to some problems

\subsubsection*{ 9}

Here is a tree representation of a graph and we want to obtain a nice tree of this graph.


We are starting from the root and it's the same node for the two trees.

In the tree on the left, each son of the root looses a letter and wins another one.

 This can be done with a join (to have two branches), a introduce (to loose a letter) and a forget (to win a letter). 
 
 So we obtain the good nodes in the nice tree.

We can make some similar steps and we obtain all the nodes of the starting tree.

It is not over because in a nice tree the leafs have to contain a single letter. But it is easy  to obtain with some introduces.

Finally we obtain a nice tree of the previous graph. 

We will use it in the next situation.



\subsubsection*{ 10-11}




The k-color problem is a well known problem where we want to know if we can color a graph with k colors so that two neighbors never have the same color.

We have seen two concepts before and each of them is usuful here : 


In other words, treewidth tells us if it is possible and nice tree gives us a way to do it.

It is a good moment to study an example : we want to color this graph we have seen before and for which we have its treewidth and a nice tree.


\subsubsection*{ 12}



the biggest bag contains 3 letters so the treewidth is 2 so we need 3 colors to solve the problem.

We can start from the "a" leaf : for example it could be colored with red.

when we make an introduce we add a color but when we make a forget we can reuse the color of the letter which is gone because we know that this letter never come back latter : this is one of the property of a tree representation.

So this alternation of forget and introduce insures that we can make the coloration with 3 colors.


Finally, the nice tree is colored with 3 colors and we can color the graph with this combination.



\subsubsection*{ 13-14}





Thank you for your attention
\end{multicols}
\end{document}
