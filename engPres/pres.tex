
\documentclass[8pt]{beamer}
\usepackage[utf8]{inputenc}
\usepackage[T1]{fontenc}
\usepackage[french]{babel}


%\usetheme{Singapore} % theme
\usetheme{CambridgeUS} % theme


\titlegraphic{\includegraphics [width=\linewidth]{logos.png}}

\title[Project's name]{Graphs' decompositions\\ and \\ resolutions of combinatorial problems}
\author[My name]{\small {Stéphane Secouard\\ \footnotesize Supervised by: Florent Madeleine}} % auteur
\institute[Caen University]{\textbf {Caen University - Computer science }}
\date{25 october 2016}





\usepackage{tikz}
\usetikzlibrary{shapes,snakes}
\setbeamercolor{titrell}{bg=blue,fg=white}
\setbeamercolor{textell}{bg=blue!10,fg=black}

%\useinnertheme{circles}
%\useoutertheme{smouthtree} 

%\logo{\includegraphics[height=10mm]{UNICAEN_LOGO_2015_V2_N_papier.png}}
%\AtBeginSection[]

%{

 % \begin{frame}

  %\frametitle{Sommaire}

  %\tableofcontents[currentsection, hideothersubsections, pausesubsections]

  %\end{frame} 

%}
%------------------------------------------------------------
\usetheme{Montpellier}
\usepackage[footheight=1em]{beamerthemeboxes}
 
\addfootboxtemplate{\color{black}}{\color{white}
     \hfill\insertframenumber/\inserttotalframenumber\hspace{2em}\null}
\begin{document}

%--------------------- diapo 1 ---------------------------------------
\begin{frame}

  \titlepage
\end{frame}

%---------------------- diapo 2 --------------------------------------

\begin{frame}

  \frametitle{ \textcolor{green!50!black}{Table of contents}}

  \tableofcontents

\end{frame}



%------------------------------------------------------------


%\begin{frame}

%  \tableofcontents[currentsection]

%\end{frame}

%------------------- diapo 3 -----------------------------------------
\section{Tree decomposition}

\subsection{definition}
\begin{frame}
\frametitle{\textcolor{green!50!black}{Tree decomposition}}  



    \begin{definition}[Graph decomposition]
  A tree T is a decomposition of a graph G  when its vertices are arranged satisfying the following properties :
  \begin{enumerate}
  \item If u and v are netighbors in G, then there is a bag of T containing both of them.
  \item For every vertex v of G, the bags of T containing v form a connected subtree
  \end{enumerate}
\end{definition}

    

\begin{columns}
\begin{column}{0.46\textwidth}
  \begin{tikzpicture}[scale=0.8]
  \uncover<2->{ \node[draw,rectangle] (G) at (-1,1.5) {\textbf{(G)}};}
  \uncover<2-> {\node[draw,circle] (1) at (0,2) {1};}
  \uncover<2-> {\node[draw,circle] (2) at (0,1) {2};}
  \uncover<2-> {\node[draw,circle] (3) at (1,1.5) {3};}
  \uncover<2-> {\node[draw,circle] (4) at (2,2) {4};}
  \uncover<2-> {\node[draw,circle] (5) at (2,1) {5};}
  \uncover<2-> {\node[draw,circle] (6) at (3,2) {6};}
  \uncover<2-> {\node[draw,circle] (7) at (3,1) {7};}
  \uncover<2-> {\node[draw,circle] (8) at (4.5,2) {8};}
  \uncover<2-> {\node[draw,circle] (9) at (4.5,1.3) {9};}
  \uncover<2-> {\node[draw,circle] (10) at (4,0.5) {{\tiny 10}};}
  \uncover<2-> {\node[draw,circle] (11) at (4.8,0.5) {{\tiny 11}};}
  \uncover<2->{\draw(1)--(3)--(2) (3)--(4)--(5)--(3) (4)--(6)--(5)--(7)--(4) (6)--(8)--(9)--(6)--(7)--(10)--(11);}
\end{tikzpicture}
\end{column}
\begin{column}{0.54\textwidth}
  \begin{tikzpicture}[scale=1]
    \uncover<2->{\draw[thick](-1,0)--(-1,2);}
    \uncover<2-> {\node[draw,rectangle](T1) at (0,2){\textbf{(T)}};}
  \only<3> \node[draw,ellipse](a) at (0,0){1,3};
  \only<3> \node[draw,ellipse](b) at (2,0){2,3};
  \only<3> \node[draw,ellipse](c) at (1,1){3,4,5};
  \only<3> \node[draw,ellipse](d) at (3,2){4,5,6,7};
  \only<3> \node[draw,ellipse](e) at (5,2){6,8,9};
  \only<3> \node[draw,ellipse](f) at (4,1){7,10};
  \only<3> \node[draw,ellipse](g) at (4.3,0){10,11};
  \only<3>\draw(a)--(c)--(b) (c)--(d)--(e) (d)--(f)--(g);
  

  \only<4>\node[draw,ellipse](h) at (1.5,1.5){4,5,6,7,8,9};
  \only<4>\node[draw,ellipse](i) at (0,0){1,2,3};
  \only<4>\node[draw,ellipse](j) at (3,0){7,10,11};
  \only<4>\draw(j)--(h)--(i);

  \only<5>\node[draw,ellipse](k) at (2,1){1,2,3,4,5,6,7,8,9,10,11};
  
  \end{tikzpicture}
  
\end{column}
\end{columns}



\end{frame}



%-------------------------------------------------------------

\end{document}

\begin{frame}
  \begin{tabular}{ll}
\frametitle{ \textcolor{green!50!black}{Un graphe}}
\begin{tikzpicture}
  \node[draw,circle] (a) at (1,3) {a};
  \node[draw,circle] (b) at (0,2) {b};
  \node[draw,circle] (c) at (2,2) {c};
  \node[draw,circle] (d) at (4,2) {d};
  \node[draw,circle] (e) at (0,0) {e};
  \node[draw,circle] (f) at (2,0) {f};
  \node[draw,circle] (g) at (4,0) {g};
  \node[draw,circle] (h) at (6,0) {h};
  \draw (c)--(a)--(b)--(e)--(f)--(c)--(b);
  \draw (c)--(d)--(f)--(g)--(d)--(h)--(g);
\end{tikzpicture}
\\\hline
\begin{tikzpicture}
  \node[draw,ellipse] (cdf) at (2,2) {c,d,f};
  \node[draw,ellipse] (bcf) at (1,1) {b,c,f};
  \node[draw,ellipse] (abc) at (0,0) {a,b,c};
  \node[draw,ellipse] (bef) at (2,0) {b,e,f};
  \node[draw,ellipse] (dfg) at (3,1) {d,f,g};
  \node[draw,ellipse] (dgh) at (4,0) {d,g,h};
  \draw (abc)--(bcf)--(cdf)--(dfg)--(dgh);
  \draw (bef)--(bcf);
\end{tikzpicture}

\end{tabular}

\begin{footnotesize}
\end{footnotesize}
\end{center}
\end{frame}

%-------------------------------------------------------------
\subsection{Treewidth}

\begin{frame}
  \begin{center}
\frametitle{ \textcolor{green!50!black}{Un arbre}}

\begin{tikzpicture}
  \node[draw,ellipse] (cdf) at (2,2) {c,d,f};
  \node[draw,ellipse] (bcf) at (1,1) {b,c,f};
  \node[draw,ellipse] (abc) at (0,0) {a,b,c};
  \node[draw,ellipse] (bef) at (2,0) {b,e,f};
  \node[draw,ellipse] (dfg) at (3,1) {d,f,g};
  \node[draw,ellipse] (dgh) at (4,0) {d,g,h};
  \draw (abc)--(bcf)--(cdf)--(dfg)--(dgh);
  \draw (bef)--(bcf);
\end{tikzpicture}

\begin{footnotesize}
\end{footnotesize}
\end{center}
\end{frame}


%-------------------------------------------------------------
\subsection{nicetree}

\begin{frame}
  \begin{center}
\frametitle{ \textcolor{green!50!black}{Un joli arbre}}
\begin{tikzpicture}
  \node[draw,ellipse] (a) at (1/6,0) {a};
  \node[draw,ellipse] (ab) at (2/6,1) {a,b};
  \node[draw,ellipse] (abc) at (3/6,2) {a,b,c};
  \node[draw,ellipse] (bc) at (4/6,3) {b,c};
  \node[draw,ellipse] (bcf) at (8/6,4) {b,c,f};
  \node[draw,ellipse] (cf) at (10/6,5) {c,f};
  \node[draw,ellipse] (cdf) at (14/6,6) {c,d,f};
  \node[draw,ellipse] (b) at (14/6,0) {b};
  \node[draw,ellipse] (be) at (13/6,1) {b,e};
  \node[draw,ellipse] (bef) at (12/6,2) {b,e,f};
  \node[draw,ellipse] (bf) at (11/6,3) {b,f};
  \node[draw,ellipse] (d) at (23/6,0) {d};
  \node[draw,ellipse] (dh) at (22/6,1) {d,h};
  \node[draw,ellipse] (dgh) at (21/6,2) {d,g,h};
  \node[draw,ellipse] (dg) at (20/6,3) {d,g};
  \node[draw,ellipse] (dfg) at (19/6,4) {d,f,g};
  \node[draw,ellipse] (df) at (18/6,5) {d,f};

  \draw (a)--(ab)--(abc)--(bc)--(bcf)--(bf)--(bef)--(be)--(b);
  \draw (bcf)--(cf)--(cdf)--(df)--(dfg)--(dg)--(dgh)--(dh)--(d);
\end{tikzpicture}

\begin{footnotesize}
\end{footnotesize}
\end{center}
\end{frame}


%-------------------------------------------------------------

\section{Applications}
\subsection{3-color}
\renewcommand\a{\textcolor{red}{a}}
\renewcommand\b{\textcolor{blue}{b}}
\renewcommand\c{\textcolor{green}{c}}
\renewcommand\d{\textcolor{blue}{d}}
\newcommand\e{\textcolor{green}{e}}
\newcommand\f{\textcolor{red}{f}}
\newcommand\g{\textcolor{green}{g}}
\newcommand\h{\textcolor{red}{h}}

\begin{frame}
  \begin{center}
    \frametitle{ \textcolor{green!50!black}{Un joli arbre}}
    \begin{tabular}{ll}
      \begin{tikzpicture}
  \node[draw,circle] (a) at (1,3) {a};
  \node[draw,circle] (b) at (0,2) {b};
  \node[draw,circle] (c) at (2,2) {c};
  \node[draw,circle] (d) at (4,2) {d};
  \node[draw,circle] (e) at (0,0) {e};
  \node[draw,circle] (f) at (2,0) {f};
  \node[draw,circle] (g) at (4,0) {g};
  \node[draw,circle] (h) at (6,0) {h};
  \draw (c)--(a)--(b)--(e)--(f)--(c)--(b);
  \draw (c)--(d)--(f)--(g)--(d)--(h)--(g);
  \only<5->\node[draw,circle,fill=red] (a) at (1,3) {a};
  \only<14-> \node[draw,circle,fill=blue] (b) at (0,2) {b};
  \only<16-> \node[draw,circle,fill=green] (c) at (2,2) {c};
  \only<22-> \node[draw,circle,fill=blue] (d) at (4,2) {d};
  \only<20-> \node[draw,circle,fill=green] (e) at (0,0) {e};
  \only<18-> \node[draw,circle,fill=red] (f) at (2,0) {f};
  \only<24-> \node[draw,circle,fill=green] (g) at (4,0) {g};
  \only<26-> \node[draw,circle,fill=red] (h) at (6,0) {h};

      \end{tikzpicture}
      &
\begin{tikzpicture}
  \node[draw,ellipse] (a) at (1/6,0) {a};
  \node[draw,ellipse] (ab) at (2/6,1) {a,b};
  \node[draw,ellipse] (abc) at (3/6,2) {a,b,c};
  \node[draw,ellipse] (bc) at (4/6,3) {b,c};
  \node[draw,ellipse] (bcf) at (8/6,4) {b,c,f};
  \node[draw,ellipse] (cf) at (10/6,5) {c,f};
  \node[draw,ellipse] (cdf) at (14/6,6) {c,d,f};
  \node[draw,ellipse] (b) at (14/6,0) {b};
  \node[draw,ellipse] (be) at (13/6,1) {b,e};
  \node[draw,ellipse] (bef) at (12/6,2) {b,e,f};
  \node[draw,ellipse] (bf) at (11/6,3) {b,f};
  \node[draw,ellipse] (d) at (23/6,0) {d};
  \node[draw,ellipse] (dh) at (22/6,1) {d,h};
  \node[draw,ellipse] (dgh) at (21/6,2) {d,g,h};
  \node[draw,ellipse] (dg) at (20/6,3) {d,g};
  \node[draw,ellipse] (dfg) at (19/6,4) {d,f,g};
  \node[draw,ellipse] (df) at (18/6,5) {d,f};
  \draw (a)--(ab)--(abc)--(bc)--(bcf)--(bf)--(bef)--(be)--(b);
  \draw (bcf)--(cf)--(cdf)--(df)--(dfg)--(dg)--(dgh)--(dh)--(d);


  
  \only<2-> \node[draw,ellipse] (a) at (1/6,0) {\a};
  \only<3-> \node[draw,ellipse] (ab) at (2/6,1) {\a,b};
  \only<4-> \node[draw,ellipse] (abc) at (3/6,2) {\a,b,c};
  
  \only<6-> \node[draw,ellipse] (ab) at (2/6,1) {\a,\b};
  \only<7-> \node[draw,ellipse] (abc) at (3/6,2) {\a,\b,c};
  \only<8-> \node[draw,ellipse] (bc) at (4/6,3) {\b,c};
  \only<9-> \node[draw,ellipse] (bcf) at (8/6,4) {\b,c,f};
  \only<10-> \node[draw,ellipse] (bf) at (11/6,3) {\b,f}; 
  \only<11-> \node[draw,ellipse] (bef) at (12/6,2) {\b,e,f};
  \only<12-> \node[draw,ellipse] (be) at (13/6,1) {\b,e};
  \only<13-> \node[draw,ellipse] (b) at (14/6,0) {\b};

  \only<15-> \node[draw,ellipse] (abc) at (3/6,2) {\a,\b,\c};
  \only<15-> \node[draw,ellipse] (abc) at (3/6,2) {\a,\b,\c};
  \only<15-> \node[draw,ellipse] (bc) at (4/6,3) {\b,\c};
  \only<15-> \node[draw,ellipse] (bcf) at (8/6,4) {\b,\c,f};
  \only<15-> \node[draw,ellipse] (cf) at (10/6,5) {\c,f};
  \only<15-> \node[draw,ellipse] (cdf) at (14/6,6) {\c,d,f};
    
  \only<17-> \node[draw,ellipse] (bf) at (11/6,3) {\b,\f}; 
  \only<17-> \node[draw,ellipse] (bef) at (12/6,2) {\b,e,\f};
  \only<17-> \node[draw,ellipse] (bcf) at (8/6,4) {\b,\c,\f};
  \only<17-> \node[draw,ellipse] (cf) at (10/6,5) {\c,\f};
  \only<17-> \node[draw,ellipse] (cdf) at (14/6,6) {\c,d,\f};
  \only<17-> \node[draw,ellipse] (dfg) at (19/6,4) {d,\f,g};
  \only<17-> \node[draw,ellipse] (df) at (18/6,5) {d,\f};

  \only<19-> \node[draw,ellipse] (be) at (13/6,1) {\b,\e};
  \only<19-> \node[draw,ellipse] (bef) at (12/6,2) {\b,\e,\f};
  
  \only<21-> \node[draw,ellipse] (cdf) at (14/6,6) {\c,\d,\f};
  \only<21-> \node[draw,ellipse] (df) at (18/6,5) {\d,\f};
  \only<21-> \node[draw,ellipse] (dfg) at (19/6,4) {\d,\f,g};
  \only<21-> \node[draw,ellipse] (dg) at (20/6,3) {\d,g};
  \only<21-> \node[draw,ellipse] (dgh) at (21/6,2) {\d,g,h};
  \only<21-> \node[draw,ellipse] (dh) at (22/6,1) {\d,h};
  \only<21-> \node[draw,ellipse] (d) at (23/6,0) {\d};

  \only<23-> \node[draw,ellipse] (dfg) at (19/6,4) {\d,\f,\g};
  \only<23-> \node[draw,ellipse] (dg) at (20/6,3) {\d,\g};
  \only<23-> \node[draw,ellipse] (dgh) at (21/6,2) {\d,\g,h};

  \only<25-> \node[draw,ellipse] (dh) at (22/6,1) {\d,\h};
  \only<25-> \node[draw,ellipse] (dgh) at (21/6,2) {\d,\g,\h};

  

  
  

\end{tikzpicture}
\\\end{tabular}

\begin{footnotesize}
\end{footnotesize}
\end{center}
\end{frame}

\end{document}
 
