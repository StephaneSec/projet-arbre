\documentclass[a4paper, 11pt,french]{article}
%\usepackage[dvips,top=1.5cm, bottom=1.5cm, left=1.5cm, right=1.5cm]{geometry}
\usepackage{PreambuleXelatex}
\usepackage{MacroTikz}
\usepackage{MacroDoc}
\pagestyle{empty}

%\setlength{\parindent}{0em}


\begin{document}
\noindent
{\small
 \begin{minipage}{0.42\linewidth}
 Stephane Secouard,

 supervise by

 Florent Madeleine
\end{minipage}
\hfill
\begin{minipage}{0.4\linewidth}
  Graphs' decompositions and \\ resolutions of combinatorial problems
\end{minipage}}
\bigskip


\begin{center}
  {\large \fbox{ABSTRACT}}
\end{center}

\medskip
A tree is a special graph but whose structure is easier to handle in many situations than a graph. For example, it is easy to browse through all its nodes without forgetting any; Moreover, in a tree, the pre-order relationship is obvious.

Unfortunately, a graph is not in general isomorphic to a tree but may be represented by a tree where nodes are collections of nodes of the initial graph.  This type of representation defines a measure to see how a given graph may ``look like''  a tree: the treewidth of the graph.

The first goal of this project is, starting from a graph whose tree representation is known, to implement a program to solve corresponding combinatorial problems. We may mention for example the 3-coloring problem, the max clique problem or the Hamilton path problem.

The second purpose is to implement a graph decomposition calculator.

Finally,it can be considered extensions by working on the efficiency of implementations on large-size structures, or improving the shape of displayed results.
\end{document}



